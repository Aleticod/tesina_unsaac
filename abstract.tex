\thispagestyle{plain}

\newpage

\begin{center}
  \textbf{ABSTRACT}
\end{center}

\vspace{1cm}

This research focuses on the generation of a custom dataset for training the YOLOv8 object detection model, specifically for detecting personal protective equipment (PPE) in construction workers. The lack of publicly available datasets with real-world images of workers using PPE motivated the creation of this dataset, which was collected at the ``Comercio'' construction site. Using an Android smartphone, images of 5792x4344 pixels were captured and preprocessed to standardize their dimensions to 640x640 pixels for YOLOv8 training.

Three image cropping algorithms were implemented: left-side cropping, centered cropping, and overlapping cropping. Additionally, the YOLOv8 pre-trained model was used to detect individuals in the images, and regions of interest were cropped accordingly, the algorithms are in the \href{https://github.com/Aleticod/yolo_image_preprocessing}{GitHub Repository}. Images without people were filtered out using other YOLOv8 model versions, yielding a final dataset of 642 cropped images. Data augmentation was applied to double the dataset size, and the images were labeled for helmet detection using Roboflow. The final dataset, with 1284 images, they are available in Roboflow platform.

\textit{\textbf{Keywords:}} Image processing, object detection, computer vision, YOLOv8, construction safety.

\vfill

\pagebreak
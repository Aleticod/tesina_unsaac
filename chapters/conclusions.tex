\chapter*{CONCLUSIONES}
\addcontentsline{toc}{chapter}{CONCLUSIONES}

\begin{itemize}
  \item Se concluye que los modelos de detección de objetos entrenados con datos generados en entornos controlados son poco eficientes al momento de hacer uso en el campo real.
  \item Las técnicas de recorte por izquierda, centrado y superpuesto son eficientes para detectar instancias alejadas en la imagen original, sin embargo, la técnica por reconocimiento es eficiente para instancias cercanas en la imagen.
  \item Después de realizado el primer filtrddo con el modelo yolov8m preentrenado con una configuración por defecto se redujo en un total 84.5\% obteniendose 1754 imágenes de un total de 11349 imágenes recortadas.
  \item Por otro lado afirmamos que la técnica de recorte superpuesto es mas eficiente, esto debido a que se obtuvo mayor cantidad de imagenes con instancias de personas, con el cual se obtuvo 503 imágenes.
  \item Al realizar un segundo filtrado con un umbral de confianza del 90\% se redujo en un 64.4\% con respecto al primer filtrado y un 94.5\% con respecto al total de imágenes obtenidas después de los recoretes.
  \item Por último al hacer uso de un modelo preentrenado de yolov8 para realizar los filtros mediante la detección de personas en las imágenes recortadas, este confunde algunos objetos con personas, dando como resultado imágenes sin personas.
\end{itemize}
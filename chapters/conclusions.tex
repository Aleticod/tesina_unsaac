\chapter*{CONCLUSIONES}
\addcontentsline{toc}{chapter}{CONCLUSIONES}

\begin{itemize}
  \item Se concluye que los modelos de deteccion de objetos entrenados con datos generados en entornos controlados son poco eficientes al memento de hacer uso en el campo real.
  \item Las tecnicas de recorte por izquierda, centrado y superpuesto son eficientes para detectar instancias alejadas en la imagen original, sin embargo, la tecnica por reconocimiento es eficiente para instancias cercanas en la imagen.
  \item Despues de realizado el primer filtrddo con el modelo yolov8m preentrenado con una configuracion por defecto se redujo en un total 84.5\% obteniendose 1754 imagenes de un total de 11349 imagenes recortadas.
  \item Por otro lado afirmamos que la tecnica de recorte superpuesto es mas eficiente, esto debido a que se obtuvo mayor cantidad de imagenes con instancias de personas, con el cual se obtuvo 503 imagenes.
  \item Al realizar un segundo filtrado con un umbral de confiaza del 90\% se redujo en un 64.4\% con respecto al primer filtrado y un 94.5\% con respecto al total de imagenes obtenidas despues del recorte.
  \item Por ultimo al hacer uso de un modelo preentrenado de yolov8 para realizar los filtros mediante la deteccion de personas en las imagenes recortadas, este confunde algunos objetos con personas, dando como resultado imagenes sin personas.
\end{itemize}
\thispagestyle{plain}

\newpage

\begin{center}
  \textbf{RESUMEN}
\end{center}

\vspace{1cm}

Esta investigación se centra en la generación de un dataset personalizado para el entrenamiento del modelo de detección de objetos YOLOv8, específicamente para la detección de equipos de protección personal (EPP) en trabajadores de la construcción. La falta de datasets públicos disponibles con imágenes del mundo real de trabajadores usando EPP motivó la creación de este dataset, el cual fue recolectado en la obra "comercio sauna". Utilizando un celular Android, se capturaron imágenes de 2546x2145 píxeles, que fueron preprocesadas para estandarizar sus dimensiones a 640x640 píxeles para el entrenamiento del YOLOv8.

Se implementaron tres algoritmos de recorte: recorte desde la izquierda, recorte centrado y recorte con superposición. Además, se utilizó el modelo preentrenado de YOLOv8 para detectar personas en las imágenes y se recortaron las áreas de interés en consecuencia. Las imágenes sin personas fueron filtradas usando otras versiones del modelo YOLOv8, obteniendo un dataset final de 624 imágenes recortadas. Se aplicó aumentación de datos para duplicar el tamaño del dataset y las imágenes fueron etiquetadas para la detección de cascos utilizando Roboflow. El dataset final, con 1248 imágenes, se exportó en formato YOLOv8 para futuros entrenamientos.

\textit{\textbf{Palabras clave:}} Procesamiento de imágenes, detección de objetos, visión por computadora, YOLOv8, seguridad en la construcción.

\vfill

\pagebreak